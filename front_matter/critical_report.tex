\documentclass[parskip=full]{scrreprt}

\RedeclareSectionCommand[pagestyle=plain,afterskip=10pt plus 1pt]{chapter}
\setkomafont{chapter}{\mdseries\headingfont\fontsize{40}{40}\selectfont\color{black!80}}
\setkomafont{pageheadfoot}{\normalsize}

\def\pnumbox#1{#1\hspace*{8cm}}
\def\gobble#1{}
\DeclareTOCStyleEntry[
  indent=0pt,
  beforeskip=0pt,
  entryformat=\itshape,
  entrynumberformat=\textcolor{oldred},
  numwidth=2em,
  linefill=\hfill,
  pagenumberbox=\pnumbox,
  pagenumberformat=\itshape
]{tocline}{part}

% use commented values if there are no parts
\DeclareTOCStyleEntry[
  indent=2em,%0pt,
  beforeskip=-\baselineskip,%0pt,
  entrynumberformat=\textcolor{oldred},
  numwidth=2em,
  linefill=\hfill,
  pagenumberbox=\pnumbox
]{tocline}{section}

\DeclareTOCStyleEntry[
  indent=0pt,
  beforeskip=-\parskip,
  entrynumberformat=\gobble,
  entryformat=\ltseries,
  numwidth=2em,
  linefill=\hfill,
  pagenumberbox=\pnumbox,
  pagenumberformat=\ltseries
]{tocline}{subsection}

\usepackage{polyglossia}
\setdefaultlanguage{english}

\usepackage{fontspec}

\setmainfont{Source Sans Pro}[
  ItalicFont = Source Sans Pro Italic,
  BoldFont = Source Sans Pro Bold,
  BoldItalicFont = Source Sans Pro Bold Italic,
  FontFace = {lt}{n}{Source Sans Pro Light},
  FontFace = {lt}{it}{Source Sans Pro Light Italic},
  FontFace = {sb}{n}{Source Sans Pro Semibold},
  FontFace = {sb}{it}{Source Sans Pro Semibold Italic},
  Numbers = Proportional,
  Ligatures = Common
]
\DeclareRobustCommand{\ltseries}{\fontseries{lt}\selectfont}
\DeclareRobustCommand{\sbseries}{\fontseries{sb}\selectfont}
\DeclareTextFontCommand{\textlt}{\ltseries}
\DeclareTextFontCommand{\textsb}{\sbseries}

\newfontfamily\headingfont{Fredericka the Great}

\usepackage[pass]{geometry}
\newgeometry{twoside,inner=20mm,outer=40mm,top=20mm,bottom=40mm}

\usepackage{url}
\urlstyle{same}

\usepackage{microtype}
\microtypesetup{verbose=silent}

\usepackage{booktabs,array,longtable}
\newcolumntype{L}[1]{>{\raggedright\let\newline\\\arraybackslash\hspace{0pt}}p{#1}}

\usepackage{graphicx}

\usepackage{xcolor}
\definecolor{oldred}{rgb}{.8313,0,0}

\usepackage{pdfpages}

\usepackage{scrlayer-scrpage}
\pagestyle{scrheadings}
\clearscrheadfoot
\cfoot[\thepage]{\thepage}
\pagenumbering{roman}

\usepackage{enumitem}
\setlist[description]{%
  style=nextline,%
  leftmargin=2em,%
  first=\ltseries,%
  font=\normalfont%
}
\def\lyrefitem#1{\item[\lyref{#1}]}


\makeatletter

\def\@firstofthree#1#2#3{#1}
\def\@secondofthree#1#2#3{#2}
\def\@thirdofthree#1#2#3{#3}
\def\Dotfill{\leavevmode\cleaders\hb@xt@ .75em{\hss .\hss }\hfill \kern \z@}

\def\lyrefnumber#1{\expandafter\@setref\csname r@#1\endcsname\@firstofthree{#1}}
\def\lyreftitle#1{\expandafter\@setref\csname r@#1\endcsname\@secondofthree{#1}}
\def\lyrefpage#1{\expandafter\@setref\csname r@#1\endcsname\@thirdofthree{#1}}

\def\lyref#1{%
  \begingroup%
  \makebox[0pt][l]{\color{oldred}\lyrefnumber{#1}}\hspace*{2em}%
  \lyreftitle{#1}\Dotfill\lyrefpage{#1}%
  \endgroup%
}
\InputIfFileExists{../out/lilypond.ref}{}{\InputIfFileExists{../lilypond.ref}{}{}}


\newcommand\fancytitlehead{
	\headingfont%
	\fontsize{80}{80}\selectfont\textcolor{black!80}{\@ifundefined{@shortname}{\@lastname}{\@shortname}.}\\[15pt]%
	\fontsize{60}{60}\selectfont\@ifundefined{@shorttitle}{\@title}{\@shorttitle}.%
}


\def\firstname#1{\def\@firstname{#1}}
\def\lastname#1{\def\@lastname{#1}}
\def\shortname#1{\def\@shortname{#1}}
\def\shorttitle#1{\def\@shorttitle{#1}}
\def\namesuffix#1{\def\@namesuffix{#1}}
\def\instrumentation#1{\def\@instrumentation{#1}}
\def\parts#1{\def\@parts{#1}}

\firstname{\relax}
\lastname{\relax}
\namesuffix{\relax}
\instrumentation{\relax}
\parts{\relax}


\def\maketitle{%
\begin{titlepage}%
	\Large%
	{\@titlehead}%
	\vfill%
	{\strut\@firstname}\\%
	{\sbseries\color{oldred}\strut\@lastname}\\%
	{\strut\@namesuffix}%
	\vfill%
	{\sbseries\@title}\\%
	{\@subtitle}\\[\baselineskip]%
	{\itshape\@instrumentation}%
	\vfill%
	{\itshape\@parts}\hspace*{\fill}\raisebox{0pt}[0pt][0pt]{\includegraphics{ees_logo}}%
\end{titlepage}%
}


\newif\iftemplate\templatetrue
\newif\ifprintreport\printreportfalse
\directlua{
scores = {
  ob1 = "Oboe I",
  ob2 = "Oboe II",
  cor12 = "Corno I, II in Es",
  ottoni = "Clarino I, II in B\string\\newline Timpani in B–F",
  vl1 = "Violino I",
  vl2 = "Violino II",
  vla = "Viola",
  coro = "Coro",
  org = "Organo",
  b = "Bassi",
  full_score = "Full Score"
}

last_arg = arg[\string#arg]
texio.write("Last argument: " .. last_arg)
if not (scores[last_arg] == nil) then
  tex.print("\string\\def\string\\lypdfname{" .. last_arg .. "}")
  tex.print("\string\\parts{" .. scores[last_arg] .. "}")
  if (last_arg == "full_score") then
    tex.print("\string\\printreporttrue")
  end
end
}

\@ifundefined{lypdfname}{%
  \templatefalse
  \printreporttrue
  \parts{Draft}
}{\templatetrue}

\makeatother






\begin{document}
\frenchspacing

\titlehead{\fancytitlehead}
\firstname{Johann Michael}
\lastname{Haydn}
\shortname{M. Haydn}
\title{Litaniae de Venerabili Sacramento}
\shorttitle{Litaniae de V. S}
\subtitle{MH 532\\F-Pn MS-2045)}
\instrumentation{S, A, T, B (solo), S, A, T, B (coro), 2 ob, 2 cor, 2 clno, timp, 2 vl, vla, b, org}
\maketitle


\thispagestyle{empty}

\vspace*{\fill}

\raisebox{-4mm}{\includegraphics{byncsaeu}}\hspace*{1em}Wolfgang Esser-Skala, 2020

© 2020 by Wolfgang Esser-Skala. This edition is licensed under the Creative Commons Attribution-NonCommercial-ShareAlike 4.0 International License. To view a copy of this license, visit \url{http://creativecommons.org/licenses/by-nc-sa/4.0/}. 

Music engraving by LilyPond 2.18.0 (\url{http://www.lilypond.org}).\\
Front matter typeset with Source Sans Pro and Fredericka the Great.

\textit{First version, October 2020}

\vspace*{2cm}

\ifprintreport
\chapter*{Critical Report.}

This edition bases upon the autograph manuscript in the Bibliothèque nationale de France, Paris. The digital version of the manuscript is available at \url{https://gallica.bnf.fr/ark:/12148/btv1b55006732j} (siglum MS-2045).

In general, this edition closely follows the manuscript. Any changes that were introduced by the editor are indicated by italic type (lyrics, dynamics and directions), parentheses (expressive marks and bass figures) or dashes (slurs and ties). Accidentals are used according to modern conventions. Asterisks denote changes that are clarified in the detailed remarks below.\footnote{Abbreviations: A, alto; B, bass; b, basses; clno, clarion; cor, horn; Ms, manuscript; ob, oboe; org, organ; r, rest; S,~soprano; T, tenor; timp, timpani; vl, violin; vla, viola.}

\bigskip

\begin{longtable}{lll L{10cm}}
	\toprule
	\itshape Mov. & \itshape Bar & \itshape Staff & \itshape Note \\
	\midrule \endhead
	3 & 24  & timp & 7th eighth in Ms: G8 \\
	5 & 261 & vl 1 & 1st quarter in Ms: d″4 \\
	\bottomrule
\end{longtable}


This edition has been compiled and checked with utmost diligence. Nevertheless, errors and mistakes cannot be totally excluded. Please report any errors and mistakes to \url{wolfgang@esser-skala.at} or create an issue or pull request on the edition’s GitHub page \url{https://github.com/skafdasschaf/haydn-m-litaniae-mh-532}. Your help will be greatly appreciated.

\bigskip
\textit{Salzburg, October 2020\\
Wolfgang Esser-Skala}

\cleardoublepage
\chapter*{Contents.}

\begin{description}
	\lyrefitem{kyrie}
	Kyrie eleison,
	Christe eleison,
	Kyrie eleison.
	Christe, audi nos,
	Christe, exaudi nos.
	Pater de coelis, Deus, miserere nobis.
	Fili Redemptor mundi, Deus, miserere nobis.
	Spiritus Sancte, Deus, miserere nobis.
	Sancta Trinitas, unus Deus, miserere nobis.
	
	\lyrefitem{panis}
	Panis vivus, qui de coelo descendisti,
	Deus absconditus et salvator,
	frumentum electorum,
	vinum germinans virgines,
	panis pinguis et deliciae regum,
	juge sacrificium,
	oblatio munda,
	agnus absque macula,
	mensa purissima, angelorum esca,
	manna absconditum,
	memoria mirabilium Dei,
	panis supersubstantialis,
	verbum caro factum, habitans in nobis,
	hostia sancta, calix benedictionis,
	mysterium fidei,
	praecelsum et venerabile Sacramentum,
	sacrificium omnium sanctissimum,
	vere propitiatorium pro vivis et defunctis,
	coeleste antidotum,
	quo a peccatis praeservamur,
	miserere nobis.
	
	\lyrefitem{stupendum}
	Stupendum supra omnia miracula,
	Sacratissima Dominicae passionis commemoratio,
	donum transcendens omnem plenitudinem,
	memoriale praecipuum divini amoris,
	divinae affluentia largitatis,
	sacrosanctum et augustissimum mysterium,
	pharmacum immortalitatis,
	miserere nobis.
	
	\lyrefitem{tremendum}
	Tremendum ac vivificum Sacramentum,
	panis omnipotentia verbi caro factus,
	miserere nobis.
	
	\lyrefitem{incruentum}
	Incruentum sacrificium,
	cibus et conviva,
	dulcissimum convivium,
	cui assistunt Angeli ministrantes,
	Sacramentum pietatis,
	vinculum charitatis,
	offerens et oblatio,
	Spiritualis dulcedo in proprio fonte degustata,
	refectio animarum sanctarum,
	viaticum in Domino morientium,
	pignus futurae gloriae,
	miserere nobis.
	
	\lyrefitem{agnusdei}
	Agnus Dei, qui tollis peccata mundi: Parce nobis Domine.
	Agnus Dei, qui tollis peccata mundi: Exaudi nos Domine.
	Agnus Dei, qui tollis peccata mundi: Miserere nobis.
\end{description}

\cleardoublepage
\fi

\iftemplate
\includepdf[pages=-]{../out/\lypdfname.pdf}
\fi

\end{document}